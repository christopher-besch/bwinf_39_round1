\documentclass[a4paper,10pt]{scrartcl}
\usepackage[ngerman]{babel}
\usepackage[T1]{fontenc}
\usepackage[utf8x]{inputenc}
\usepackage{geometry}

% mehr pakete: mathe-modus, code-environment, geometry

\title{Aufgabe 1: \LaTeX-Dokument}
\author{Max Mustermännchen}
\date{\today}

\begin{document}
\maketitle

\section{Lösungsidee}
Die Idee der Lösung sollte hieraus vollkommen ersichtlich werden, ohne das auf die eigentliche Implementation Bezug genommen wird.

\section{Umsetzung}
Hier wird kurz erläutert, wie die Lösungsidee im Programm tatsächlich umgesetzt wurde. Hier können auch Implementierungsdetails erwähnt werden.

\section{Beispiele}
Genügend Beispiele einbinden! Eigene Beispiele sind sehr gut! Und die Beispiele sollte diskutiert werden.

\section{Quellcode (ausschnittsweise)}
Unwichtige Teile des Programms müssen hier nicht abgedruckt werden.

\end{document}
